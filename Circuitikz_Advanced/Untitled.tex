\documentclass{article}
\usepackage{tikz}
\usepackage{circuitikz}
\begin{document}
	% Lines
	
	% print line with open ends
	\begin{figure}[h!]
	\begin{circuitikz}
		\draw (-1,0) to[short,o-o] (1,0);
	\end{circuitikz}
	\end{figure}
	
	% print line with closed end on the left and without end on the right
	\begin{figure}[h!]
	\begin{circuitikz}
  		\draw (-1,0) to[short,*-] (1,0);
	\end{circuitikz}
	\end{figure}
	% you can create different lines by using simple syntax like o-, -o, or o-o
	
	% Monopoles - the most basic circuitikz components
	
	\begin{figure}[h!]
	\begin{circuitikz}
  		\draw (-1,0) to[short,o-o] (1,0);
  		\draw (0,0) to[short] node[ground] {GND} (0,-1); % you can add names to be displayed next to the particular node
	\end{circuitikz}
	\end{figure}
	
	% Current arrows
	
	% basic example
	\begin{figure}[h!]
	\begin{circuitikz}
  		\draw (0,0) to[R,i=$i_1$] (2,0);
	\end{circuitikz}
	\end{figure}
	
	% switched arrow direction
	\begin{figure}[h!]
	\begin{circuitikz}
  		\draw (0,0) to[R,i<=$i_1$] (2,0);
	\end{circuitikz}
	\end{figure}
	
	% place element description above
	\begin{figure}[h!]
	\begin{circuitikz}
  		\draw (0,0) to[R,i^=$i_1$] (2,0);
	\end{circuitikz}
	\end{figure}
	
	% place element description below
	\begin{figure}[h!]
	\begin{circuitikz}
  		\draw (0,0) to[R,i_=$i_1$] (2,0);
	\end{circuitikz}
	\end{figure}
 
	% combinations of arrow direction and description placement
	% left above
	\begin{figure}[h!]
	\begin{circuitikz}
  		\draw (0,0) to[R,i<^=$i_1$] (2,0);
	\end{circuitikz}
	\end{figure}
	
	% right below
	\begin{figure}[h!]
	\begin{circuitikz}
  		\draw (0,0) to[R,i_>=$i_1$] (2,0);
	\end{circuitikz}
	\end{figure}
\end{document}